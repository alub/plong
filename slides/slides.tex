\documentclass{beamer}

% Theme and page layout
\usetheme{gapz}
\setbeamertemplate{navigation symbols}{}

% Fonts
\usepackage{fontspec}
\defaultfontfeatures{Mapping=tex-text,Scale=MatchLowercase,Numbers=Lining}
%\setsansfont[
%	ItalicFont = Whitney MediumItalic,
%	BoldFont = Whitney Semibold,
%	BoldItalicFont = Whitney SemiboldItalic,
%	SmallCapsFont = Whitney MediumSC
%]{Whitney Medium}

% Language
\usepackage{polyglossia}
\setdefaultlanguage{english}

% Graphics
\usepackage{graphicx}
\graphicspath{{./images/}}
\usepackage{multicol}

\title{Three-dimensional modeling and printing project}
\subtitle{Third-year project}
\author[V. Duvert, A. Lubineau, C. Naud, J. Packer, F. Ribon]{\scriptsize
Vincent~Duvert \\ Antoine~Lubineau \\ Caroline~Naud \\ James~Packer \\ Florian~Ribon}
\date{from January 23 to March 16, 2012}

\titlegraphic{\includegraphics[width=4cm]{inp-enseeiht}}

\begin{document}

\frame{\titlepage}

\section{Presentation of the project}

\subsection{Clients}
\begin{frame}
	\frametitle{}
	
	\begin{block}{The \textsc{VORTEX} team of IRIT}
		\begin{itemize}
		\item \textsc{IRIT} : a research institute in computer science from Toulouse
		\item \textsc{VORTEX} : Visual Objects : from Reality To EXpression
		\item skills in the areas of graphic computing,  computer vision, artificial intelligence and networks
		\item recently acquired an Ultimaker 3D printer on which she has no control over
		\end{itemize}
    \end{block}
    
    \begin{center}
		\includegraphics[width=4cm]{irit}
	\end{center}
    
\end{frame}

\subsection{Resources}
\begin{frame}
	\frametitle{}
	
	\begin{block}{The project team}
		\begin{itemize}
		\item Five final year students at ENSEEIHT, Computing and Applied Mathematics department
		\item several followed the “multimedia” option in third year
		\end{itemize}
    \end{block}
    
    \begin{block}{The material resources}
    	\begin{itemize}
		\item room F117 in building F at ENSEEIHT
    	\item Ultimaker 3D printer with a roll of PLA plastic
    	\item computer with dual-touch screen Acer T231H
    	\item computer rooms of the ENSEEIHT and personnal computer
		\end{itemize}
    \end{block}
      
\end{frame}

\begin{frame}
	\frametitle{Ultimaker 3D printer}
	
	\begin{columns}[t]
  	\begin{column}{5cm}
  		\begin{figure}
		\includegraphics[width=4cm]{Ultimaker}	
		\end{figure}
  	\end{column}
  
  	\begin{column}{5cm}
  		\begin{block}{Process description}
  		\begin{enumerate}
  		\item The mesh is sliced to create instructions
  		\item The instructions are sent to the printer
  		\item The object is created by laying down successive layers of material
  		\end{enumerate}
 	 	\end{block}   
  	\end{column}
 	\end{columns}  
\end{frame}


\subsection{Objectives}
\begin{frame}
	\frametitle{The main client's requests}
	\begin{block}{Main goals of the project} 
	\begin{itemize}	
		\item development of a fully open-source software with graphical interface that allows to import, model and deform virtual 3D objects, preferably using a dualtouch screen
		\item export of the corresponding meshes to the printer to finally print it as realistically as possible.
	\end{itemize}
    \end{block}
    
    \begin{block}{The final users}
    	\begin{itemize}
		\item The \textsc{VORTEX} team
		\item Some artists from Artilect for example (or artists students)
		\end{itemize}
    \end{block}
    
\end{frame}

\section{Project management}

\begin{frame}
	\frametitle{The tools}
Project manager : Caroline \textsc{NAUD}
	\begin{block}{}
    \begin{itemize}
		\item a web based project management system : Chiliprojet (Gantt charts, calendars, documents, Wiki, code deposit, notifications ...)
		\item a technical supervisor from \textsc{Airbus} : Lionel \textsc{CREMEL}
		\item regular meetings with the clients and the supervisor (at least once a week)
	\end{itemize}
	\end{block}
\end{frame}

\begin{frame}
	\frametitle{Extract of our Gantt chart}

    \begin{center}
		\includegraphics[width=8cm]{VCycle}	
	\end{center}
\end{frame}

\begin{frame}
	\frametitle{The V cycle management strategy}

    
\end{frame}

\begin{frame}
	\frametitle{The macroscopic planning}

    \begin{center}
		\includegraphics[width=9cm]{PlanningMacroscopique}	
	\end{center}
	
	\begin{block}{}
    \begin{itemize}
		\item Mx : milestones to respect : mainly documents to provide
		\item BC: Business Case
		\item SOW : Statement Of Work
		\item BRD \& ARD : the application design 
	\end{itemize}
	\end{block}
	
\end{frame}


\section{Architecture of the project}

\begin{frame}
	\frametitle{A four steps application ... }
	
	\begin{block}{}
	\begin{itemize}
	\item The modeller : to import/create STL and PLY meshes and to pre-process them.
	\item The Gcode Generator : to generate the printer's instructions.
	\item The printer's pilote : to send the instructions including special parameters such as the temperature.
	\item The printer : to create the final object. It must be monitored.
	\end{itemize}
	\end{block}
	
\end{frame}


\begin{frame}
	\frametitle{From the modeller to the printer ...}

    \begin{center}
		\includegraphics[width=8cm]{ARD1}	
	\end{center}
	
\end{frame}

\section{The main specifications}

\subsection{About the application}
\begin{frame}
	\frametitle{}
	 \begin{block}{The application characteristics}
		\begin{itemize}
			\item must be fully open source
			\item must be usable on both Linux Ubuntu and Windows (if possible)
			\item must be delivered as a tarball
		\end{itemize}
    \end{block}
\end{frame}
    
\subsection{About the modeller}
\begin{frame}
	\frametitle{}
	 \begin{block}{The modeller characteristics}
		\begin{itemize}
			\item must allow the import of STL and PLY meshes
			\item must allow the export of STL meshes
			\item all the functions implemented must be run in less than a few hours
		\end{itemize}
    \end{block}
    
    \begin{block}{The interactions required on the meshes}
		\begin{itemize}
			\item 
			\item 
		\end{itemize}
    \end{block}

\end{frame}

\subsection{About the documentation}
\begin{frame}
	\frametitle{}
	 \begin{block}{The documentation required}
		\begin{itemize}
			\item a user documentation : to use the application properly just by reading it. It should include the limits of the application.
			\item a technical documentation : so that the client or an other team can continue the work if needed
		\end{itemize}
    \end{block}    

\end{frame}

\section{Our technical solutions and results}

\subsection{Choice of the modeller}
\begin{frame}
	\frametitle{}
		--> Implementation of a modeller from scratch excluded \\
		--> Choice reduced to Blender or Meshlab
    \begin{columns}[t]
  	\begin{column}{5cm}
  		\begin{block}{Meshlab}
  		Advantages :
  		\begin{itemize}
  		\item 
  		\item 
  		\item
  		\end{itemize}
  		
  		Drawbacks :
  		\begin{itemize}
  		\item 
  		\item 
  		\item
  		\end{itemize}
 	 	\end{block}   
  	\end{column}
  
  	\begin{column}{5cm}
  		\begin{block}{Blender (the chosen one)}
  		Advantages :
  		\begin{itemize}
  		\item 
  		\item 
  		\item
  		\end{itemize}
  		
  		Drawbacks :
  		\begin{itemize}
  		\item 
  		\item 
  		\item
  		\end{itemize}
 	 	\end{block}   
  	\end{column}
 	\end{columns}  
    
\end{frame}

\subsection{Mesh correction}
\begin{frame}
	\frametitle{}

    \begin{block}{Manifold correction}
		\begin{itemize}
			\item manifold checking and hole filling already included in Blender
			\item some bad results
		\end{itemize}
    \end{block}
\end{frame}

\subsection{Modifications in Blender's interface}
\begin{frame}
	\frametitle{}

    \begin{block}{A new screen for Blender}
Possibility in Blender to manually modify the interface and to register it as the default one.
Changes operated :
	\begin{itemize}
	\item 
	\item
	\item
	\end{itemize}
    \end{block}
    
    \begin{block}{A new panel}
    \begin{itemize}
	\item Addition on a new panel named "Mesh verification" thank to the Blender Python API
	\item Insertion of buttons appearing conditionally and calling the mesh corrections implemented
	\end{itemize}
    \end{block}
\end{frame}

\subsection{Calibration of the printer}
\begin{frame}
	\frametitle{}

    \begin{block}{}
    \end{block}
\end{frame}

\section{Conclusions and thanks}
\begin{frame}
	\frametitle{}

    \begin{block}{}
    \end{block}
\end{frame}

\begin{frame}
	\frametitle{}

    \begin{center}
    \Large{Thank you for your attention !}
    \end{center}
\end{frame}
	
\end{document}
